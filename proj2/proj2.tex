\documentclass[hidelinks, twocolumn, a4paper, 11pt]{article}
\usepackage[left=1.4cm, top=2.3cm, text={18.2cm,25.2cm}]{geometry}
\usepackage[utf8]{inputenc}
\usepackage[czech]{babel}
\usepackage[IL2]{fontenc}
\usepackage{color}
\usepackage{hyperref}
\usepackage{times}
\usepackage{amsthm}
\usepackage{amsmath}
\usepackage{amsfonts}
\theoremstyle{definition}
\newtheorem{theorem_def}{Definice}
\newtheorem{theorem_veta}{Věta}

%\color{red}

\begin{document}

\begin{titlepage}
    
    \begin{center}
        {\Huge \textsc{Vysoké učení technické v Brně}\\[0.5em]}
        {\huge \textsc{Fakulta informačních technologií}\\}
        \vspace{\stretch{0.382}}
        {\LARGE Typografie a publikování\,--\,2. projekt}\\[0.4em]
        {\LARGE Sazba dokumentů a matematických výrazů \\}
        \vspace{\stretch{0.618}}
    \end{center}

    {\Large 2023 \hfill David Svaty (xsvaty01)}

\end{titlepage}

\label{pag:1}
\section*{Úvod}
V této úloze si vyzkoušíme sazbu titulní strany, matematických vzorců,
prostředí a dalších textových struktur obvyklých pro technicky zaměřené texty\,--\,například 
Definice \ref{def:1} nebo rovnice \eqref{eq:3} na straně \pageref{pag:1}.
Pro vytvoření těchto odkazů používáme kombinace příkazů
\verb|\label|, \verb|\ref|, \verb|\eqref| a \verb|\pageref|. Před odkazy patří nezlomitelná mezera.
Pro zvýrazňování textu jsou zde několikrát použity příkazy \verb|\verb| a \verb|\emph|.

Na titulní straně je použito prostředí \verb|titlepage| a sázení nadpisu podle optického
středu s využitím \emph{přesného} zlatého řezu. Tento postup byl probírán na přednášce.
Dále jsou na titulní straně použity čtyři různé velikosti písma a 
mezi dvojicemi řádků textu je použito odřádkování se zadanou relativní velikostí 
0,5\,em a 0,4\,em\footnote[1]{Nezapomeňte použít správný typ mezery mezi číslem a jednotkou.}.

\section{Matematický text}
V této sekci se podíváme na sázení matematických symbolů a výrazů v plynulém textu 
pomocí prostředí \verb|math|. Definice a věty sázíme pomocí příkazu \verb|\newtheorem| s 
využitím balíku \verb|amsthm|. Někdy je vhodné použít konstrukci \verb|${}$| nebo \verb|\mbox{}|, 
která říká, že (matematický) text nemá být zalomen. 


\begin{theorem_def}
    \label{def:1}
    Zásobníkový automat \emph{(ZA) je definován jako sedmice tvaru $A = (Q,\Sigma,\Gamma,\delta, q_0, Z_0, F)$, kde}:     

    \begin{itemize}
        \item $Q$ \emph{je konečná množina} vnitřních (řídicích) stavů, 
        \item $\Sigma$ \emph{je konečná} vstupní abeceda, 
        \item $\Gamma$ \emph{je konečná} zásobníková abeceda, 
        \item $\delta$ \emph{je} přechodová funkce $Q \times (\Sigma \cup \{\epsilon\}) \times \Gamma \rightarrow 2^{Q\times\Gamma^\ast}$,
        \item $q_0 \in Q$ je počáteční stav, $Z_0 \in \Gamma$ \emph{je} startovací symbol zásobníku a $F \subseteq Q$ \emph{je množina} koncových stavů. 
    \end{itemize}
\end{theorem_def}


Nechť $P = (Q,\Sigma,\Gamma,\delta, q_0, Z_0, F)$ je ZA. 
\emph{Konfigurací} nazveme trojici $(q,w,\alpha) \in Q \times \Sigma^\ast \times \Gamma^\ast$, 
kde $q$ je aktuální stav vnitřního řízení, 
$w$ je dosud nezpracovaná část vstupního řetězce a
$\alpha = Z_{i_1}Z_{i_2}\dots Z_{i_k}$ je obsah zásobníku.

\subsection{Podsekce obsahující definici a větu}

%TODO: oprav
\begin{theorem_def}
    Řetězec $w$ nad abecedou $\Sigma$ je přijat ZA \emph{A~jestliže $(q_0, w, Z_0) \underset{A}{\overset{*}{\vdash}} (q_F,\epsilon,\gamma)$ 
    pro nějaké $\gamma \in \Gamma^\ast$ a $q_F \in F$. 
    Množina $L(A) = \{ w \mid w \mbox{ je přijat ZA } A \} \subseteq \Sigma^\ast$ je} jazyk přijímaný ZA $A$.
\end{theorem_def}

\begin{theorem_veta}
    \emph{Třída jazyků, které jsou přijímány ZA, odpovídá} bezkontextovým jazykům.
\end{theorem_veta}

\section{Rovnice}
Složitější matematické formulace sázíme mimo plynulý text pomocí prostředí \verb|displaymath|. 
Lze umístit i několik výrazů na jeden řádek, ale pak je třeba tyto vhodně oddělit, 
například příkazem \verb|\quad|. 
\[ 
1^{2^3} \neq \Delta_{\Delta_{\Delta^3}^2}^1 \quad
y^{11}_{22} - \sqrt[9]{x+\sqrt[7]{y}} \quad 
x > y_1 \leq y^2    
\]
V rovnici (\ref{eq:2}) jsou využity tři typy závorek s různou \emph{explicitně} definovanou velikostí. 
Také nepřehlédněte, že nasledující tři rovnice mají zarovnaná rovnítka, 
a použijte k~tomuto účelu vhodné prostředí. 
\begin{eqnarray}
    -\cos^2\beta & = & \frac{\frac{\frac{1}{x} + \frac{1}{3}}{y}+100}{\prod\limits_{j=2}^8 q_j} \label{eq:1} \\
    \biggl( \Bigl\{ b \star \bigl[3 \div 4 \bigl]\;\circ\;a \Bigl\}^{\frac{2}{3}} \biggl) & = & \log_{10} x \label{eq:2}\\
    \int_a^b f(x)\,\mathrm{d}x & = & \int_c^d f(y)\,\mathrm{d}y \label{eq:3}
\end{eqnarray}
V této větě vidíme, jak vypadá implicitní vysázení limity $\lim_{m \to \infty} f(m)$ v normálním odstavci textu. 
Podobně je to i s dalšími symboly jako $ \bigcup_{N \in \mathcal{M}} N$ či $\sum_{i=1}^{m} x_i^2$. 
S vynucením méně úsporné sazby příkazem \verb|\limits| budou vzorce vysázeny v podobě $ \lim\limits_{m \to \infty} f(m)$ a $ \sum\limits_{i=1}^m x_i^4 $.

\section{Matice}
Pro sázení matic se velmi často používá prostředí \verb|array| a závorky (\verb|\left|, \verb|\right|). 
\[
    \mathbf{B} = 
    \left| \begin{array}{cccc}
        b_{11} & b_{12} & \cdots & b_{1n} \\
        b_{21} & b_{22} & \cdots & b_{2n} \\
        \vdots & \vdots & \ddots & \vdots \\ 
        b_{m1} & b_{m2} & \cdots & b_{mn}
    \end{array} \right|
    =
    \left| \begin{array}{cc}
        t & u \\
        v & w
    \end{array} \right|
    =
    tw - uv
\]
\[
    \mathbb{X} = \mathbf{Y} \Longleftrightarrow 
    \left[ \begin{array}{ccc}
        & \Omega + \Delta & \hat{\psi} \\
        \vec{\pi} & \omega & 
    \end{array} \right]
    \neq 42
\]

Prostředí \verb|array| lze úspěšně využít i jinde, například na pravé straně následující rovnice. 
Kombinační číslo na levé straně vysázejte pomocí příkazu \verb|\binom|.

\[
    \binom{n}{k}
    =
    \left\{
    \begin{array}{c l}
        0 & \text{pro } k<0 \\
        \frac{n!}{k!(n-k)!} & \text{pro } 0 \leq k \leq n \\
        0 & \text{pro } k>0
    \end{array} \right.
\]

\end{document}



