\documentclass[hidelinks, twocolumn, a4paper, 11pt]{article}
\usepackage[left=1.4cm, text={18.2cm,25.2cm}, top=2.3cm]{geometry}
\usepackage[utf8]{inputenc}
\usepackage[czech]{babel}
\usepackage[IL2]{fontenc}
\usepackage{color}
\usepackage{hyperref}
\usepackage{times}
\usepackage{amsthm}
\theoremstyle{definition}
\newtheorem{theorem_def}{Definice}
\newtheorem{theorem_veta}{Věta}



\begin{document}

\begin{titlepage}
    
    % TODO: velikosti pisma 
    \begin{center}
        \Huge
        \textsc{Vysoké učení technické v Brně \\[0.5em]
        {\huge Fakulta informačních technologií}} \\
        \vspace{\stretch{0.382}}
        {\LARGE Typografie a publikování -- 2. projekt \\[0.4em]
        Sazba dokumentů a matematických výrazů} \\
        \vspace{\stretch{0.618}}
    \end{center}

    {\LARGE 2023 \hfill David Svaty (xsvaty01)}

\end{titlepage}

\section*{Úvod}
V této úloze si vyzkoušíme sazbu titulní strany, matematických vzorců, 
prostředí a dalších textových struktur obvyklých pro technicky zaměřené texty --
například Definice nebo rovnice na straně. 
Pro vytvoření těchto odkazů používáme kombinace příkazů
\verb|\label|, \verb|\ref|, \verb|\eqref| a \verb|\pageref|. Před odkazy patří nezlomitelná mezera. 
Pro zvýrazňování textu jsou zde několikrát použity příkazy \verb|\verb| a \verb|\emph|.

Na titulní straně je použito prostředí \verb|titlepage| a sázení nadpisu podle optického
středu s využitím \textsl{přesného} zlatého řezu. Tento postup byl probírán na přednášce.
Dále jsou na titulní straně použity čtyři různé velikosti písma a 
mezi dvojicemi řádků textu je použito odřádkování se zadanou relativní velikostí 
0,5\,em a 0,4\,em\footnote[1]{Nezapomeňte použít správný typ mezery mezi číslem a jednotkou.}.

\section{Matematický text}

V této sekci se podíváme na sázení matematických symbolů a výrazů v plynulém textu 
pomocí prostředí \verb|math|. Definice a věty sázíme pomocí příkazu \verb|\newtheorem| s 
využitím balíku \verb|amsthm|. Někdy je vhodné použít konstrukci \verb|${}$| nebo \verb|\mbox{}|, 
která říká, že (matematický) text nemá být zalomen. 


\begin{theorem_def}
    Zásobníkový automat \textsl{(ZA) je definován jako sedmice tvaru $A = (Q,\Sigma,\Gamma,\delta, q_0, Z_0, F)$, kde}:     
\end{theorem_def}

\begin{itemize}
    \item $Q$ \textsl{je konečná množina} vnitřních (řídicích) stavů, 
    \item $\Sigma$ \textsl{je konečná} vstupní abeceda, 
    \item $\Gamma$ \textsl{je konečná} zásobníková abeceda, 
    \item $\delta$ \textsl{je} přechodová funkce $Q \times (\Sigma \cup \{\epsilon\}) \times \Gamma \rightarrow 2^{Q\times\Gamma^\ast}$,
    \item $q_0 \in Q$ je počáteční stav, $Z_0 \in \Gamma$ \textsl{je} startovací symbol zásobníku a $F \subseteq Q$ \textsl{je množina} koncových stavů. 
\end{itemize}

Nechť $P = (Q,\Sigma,\Gamma,\delta, q_0, Z_0, F)$ je ZA. 
\textsl{Konfigurací} nazveme trojici $(q,w,\alpha) \in Q \times \Sigma^\ast \times \Gamma^\ast$, 
kde $q$ je aktuální stav vnitřního řízení, 
$w$ je dosud nezpracovaná část vstupního řetězce a
$\alpha = Z_{i_1}Z_{i_2}\dots Z_{i_k}$ je obsah zásobníku.

\subsection{Podsekce obsahující definici a větu}

%TODO: oprav
\begin{theorem_def}
    Řetězec $w$ nad abecedou $\Sigma$ je přijat ZA \textsl{A jestliže $(q_0, w, Z_0) (q_F,\epsilon,\gamma)$ 
    pro nějaké $\gamma \in \Gamma^\ast$ a $q_F \in F$. 
    Množina $L(A) = \{ w | w je přijat ZA A \} \subseteq \Sigma^\ast$ je} jazyk přijímaný ZA $A$.
\end{theorem_def}

\begin{theorem_veta}
    \textsl{Třída jazyků, které jsou přijímány ZA, odpovídá} bezkontextovým jazykům.
\end{theorem_veta}

\end{document}