\documentclass[a4paper,11pt, hidelinks]{article}
\usepackage[left=2cm, top=3cm, text={17cm,24cm}]{geometry}
\usepackage{times}
\usepackage[czech]{babel}
\usepackage[utf8]{inputenc}
\usepackage{multirow}
\usepackage{subcaption}
\usepackage{hyperref}
\usepackage[ruled,onelanguage,czech,commentsnumbered,longend]{algorithm2e}

\begin{document}
    
\begin{titlepage}
    \begin{center}
        {\Huge \textsc{Vysoké učení technické v Brně}\\[0.5em]}
        {\huge \textsc{Fakulta informačních technologií}\\}
        \vspace{\stretch{0.382}}
        {\LARGE Typografie a publikování\,--\,3. projekt}\\[0.4em]
        {\Huge Tabulky a obrazky \\}
        \vspace{\stretch{0.618}}
    \end{center}

    {\Large \today \hfill David Svaty (xsvaty01)}
\end{titlepage}

\label{page:1}

\section{Úvodní strana}
\label{sec:1}
Název práce umístněte do zlatého řezu a nezapomeňte uvést \uv{dnešní} (today) datum a vaše jméno a příjmení.

\section{Tabulky}
\label{sec:2}
Pro sázení tabulek můžeme použít prostředí \texttt{tabbing} nebo prostředí \texttt{tabular}.

\subsection{Prostředí \texttt{tabbing}}
Při použití \texttt{tabbing} vypadá tabulka následovně:
\begin{tabbing}
    \textbf{Ovoce} \hspace{16mm} \= \textbf{Cena} \hspace{4mm} \= \textbf{Množství} \\
    Jablka \> 25,90 \> 3 kg \\
    Hrušky \> 27,40 \> 2,5 kg \\
    Vodní melouny \> 35,- \> 1 kus \\
\end{tabbing}
\label{tab:1}
Toto prostředí se dá také použít pro sázení algoritmů, ovšem vhodnější je použít prostředí \texttt{alogorithm} nebo \texttt{algorithm2e} (viz sekce 3).

\subsection{Prostředí \texttt{tabular}}
Další možností, jak vytvořit tabulku je prostředí \texttt{tabular}. Tabulky pak budou vypadat takto\footnote[1]{Kdyby byl problém s \texttt{cline}, zkuste se podívat třeba sem: http://www.abclinuxu.cz/tex/poradna/show/325037.}:

\begin{table}[h]
\catcode`\-=12
\centering
\begin{tabular}{|c|c|c|}
    \hline
    \multirow{2}{*}{\textbf{Měna}} & \multicolumn{2}{|c|}{\textbf{Cena}} \\
    \cline{2-3} %TODO: fix
     & \textbf{nákup} & \textbf{prodej} \\
    \hline
    EUR & 22,705 & 25,242 \\
    GBP & 25,931 & 28,828 \\
    USD & 21,347 & 23,732 \\
    \hline
\end{tabular}
\caption{Tabulka kurzů k dnešnímu dni}
\label{tab:2}
\end{table}

\begin{table}[h]
\catcode`\-=12
\centering
\begin{subtable}[h]{0.085\textwidth}
    \centering
    \begin{tabular}{|c|c|}
        \hline
        \textit{A} & $\neg$\textit{A} \\
        \hline
        \textbf{P} & N \\
        \hline
        \textbf{O} & O \\
        \hline
        \textbf{X} & X \\
        \hline
        \textbf{N} & P \\
        \hline
    \end{tabular}
\end{subtable}
\;
\begin{subtable}{0.235\textwidth}
    \centering
    \begin{tabular}{|c|c|c|c|c|c|}
        \hline
        \multicolumn{2}{|c|}{\multirow{2}{*}{\textit{A} $\land$ \textit{B}}} & \multicolumn{4}{|c|}{\textit{B}} \\
        \cline{3-6}
        \multicolumn{2}{|c|}{} &\textbf{P} & \textbf{O} & \textbf{X} & \textbf{N} \\
        \hline
        \multirow{4}{*}{\textit{A}} & \textbf{P} & P & O & X & N \\
        \cline{2-6}
        & \textbf{O} & O & O & N & N \\
        \cline{2-6}
        & \textbf{X} & X & N & X & N \\
        \cline{2-6}
        & \textbf{N} & N & N & N & N \\
        \hline 
    \end{tabular}
\end{subtable}
\;
\begin{subtable}{0.235\textwidth}
    \centering
    \begin{tabular}{|c|c|c|c|c|c|}
        \hline
        \multicolumn{2}{|c|}{\multirow{2}{*}{\textit{A} $\lor$ \textit{B}}} & \multicolumn{4}{|c|}{\textit{B}} \\
        \cline{3-6}
        \multicolumn{2}{|c|}{} &\textbf{P} & \textbf{O} & \textbf{X} & \textbf{N} \\
        \hline
        \multirow{4}{*}{\textit{A}} & \textbf{P} & P & P & P & P \\
        \cline{2-6}
        & \textbf{O} & P & O & P & O \\
        \cline{2-6}
        & \textbf{X} & P & P & X & X \\
        \cline{2-6}
        & \textbf{N} & P & O & X & N \\
        \hline 
    \end{tabular}
\end{subtable}
\;
\begin{subtable}{0.235\textwidth}
    \centering
    \begin{tabular}{|c|c|c|c|c|c|}
        \hline
        \multicolumn{2}{|c|}{\multirow{2}{*}{\textit{A} $\to$ \textit{B}}} & \multicolumn{4}{|c|}{\textit{B}} \\
        \cline{3-6}
        \multicolumn{2}{|c|}{} &\textbf{P} & \textbf{O} & \textbf{X} & \textbf{N} \\
        \hline
        \multirow{4}{*}{\textit{A}} & \textbf{P} & P & O & X & N \\
        \cline{2-6}
        & \textbf{O} & P & O & P & O \\
        \cline{2-6}
        & \textbf{X} & P & P & X & X \\
        \cline{2-6}
        & \textbf{N} & P & P & P & P \\
        \hline 
    \end{tabular}
\end{subtable}
\label{tab:3}
\caption{Protože Kleeneho trojhodnotová logika už je \uv{zastaralá}, uvádíme si zde příklad čtyřhodnotové logiky}
\end{table}

\newpage %TODO: odstran 

\section{Algoritmy}
Pokud budeme chtít vysázet algoritmus, můžeme použít prostředí \texttt{algorithm}\footnote[2]{Pro nápovědu, jak zacházet s prostředím \texttt{algorithm}, můžeme zkusit tuhle stránku: \\
http://ftp.cstug.cz/pub/tex/CTAN/macros/latex/contrib/algorithms/algorithms.pdf.}
nebo \texttt{algorithm2e}\footnote[3]{Pro \texttt{algorithm2e} zase tuhle: http://ftp.cstug.cz/pub/tex/CTAN/macros/latex/contrib/algorithm2e/doc/algorithm2e.pdf.}.
Příklad použití prostředí \texttt{algorithm2e} viz Algoritmus 1.

\begin{algorithm}
\caption{F{\footnotesize AST}SLAM}
\SetAlgoNoLine

\SetNlSty{}{}{:}

\KwIn{$(X_{t-1}, u_1, z_t)$}
\KwOut{$X_t$}

\BlankLine
\nl $\overline{X_t} = X_t = 0$ \\
\nl \For(){k=1 \normalfont{to} M}{
    \nl $x_t^{[k]} = sample\_motion\_model(u_t,x_{t-1}^{[l]})$ \\
    \nl $\omega_t^{[k]} = measurement\_model(z_t,x_t^{[k]},m_{r-1})$ \\
    \nl $m_t^{[k]} = updated\_occupancy\_grid(z_t,x_t^{[k]},m_{t-1}^{[k]})$ \\
    \nl $\overline{X_t} = \overline{X_t} + \langle x_x^{[m]}, \omega_t^{[m]} \rangle$
}

\nl \For(){k=1 \normalfont{to} M}{
    \nl $\mbox{draw \emph{i} with probability} \approx \omega_t^{[i]}$ \\
}


\end{algorithm}

\end{document}